% resume.tex
% vim:set ft=tex spell:

\documentclass[10pt,letterpaper]{article}
\author{Jon Hatfield}
\title{Jon Hatfield Resume}
\usepackage[letterpaper,margin=0.75in]{geometry}
\usepackage[utf8]{inputenc}
\usepackage{mdwlist}
\usepackage[T1]{fontenc}
\usepackage{textcomp}
\usepackage{tgpagella}
%\usepackage[hidelinks]{hyperref}
\usepackage{hyperref}
\usepackage[usenames,dvipsnames,svgnames,table]{xcolor}
%\usepackage{color}
\usepackage{ifthen}
\usepackage{etoolbox}
\pagestyle{empty}
\setlength{\tabcolsep}{0em}
%\urlstyle{same}

% indentsection style, used for sections that aren't already in lists
% that need indentation to the level of all text in the document
\newenvironment{indentsection}[1]%
{\begin{list}{}%
    {\setlength{\leftmargin}{#1}}%
    \item[]%
}
{\end{list}}

% opposite of above; bump a section back toward the left margin
\newenvironment{unindentsection}[1]%
{\begin{list}{}%
    {\setlength{\leftmargin}{-0.5#1}}%
    \item[]%
}
{\end{list}}

% format two pieces of text, one left aligned and one right aligned
\newcommand{\headerrow}[2]
{\begin{tabular*}{\linewidth}{l@{\extracolsep{\fill}}r}
    #1 &
    #2 \\
\end{tabular*}}

% make "C++" look pretty when used in text by touching up the plus signs
\newcommand{\CPP}
{C\nolinebreak[4]\hspace{-.05em}\raisebox{.22ex}{\footnotesize\bf ++}}

% and the actual content starts here
\begin{document}

\begin{center}
    %{\LARGE \textbf{% name.tex
% vim:set ft=tex spell:
%
Jon Hatfield
}}
    {\huge \textbf{% name.tex
% vim:set ft=tex spell:
%
Jon Hatfield
}}
    \ifnum\pdfstrcmp{\jobname}{JonHatfieldPublicResume}=0
        % city.tex
% vim:set ft=tex spell:
%
Indianapolis, IN
 \ \textbullet \ \ % email.tex
% vim:set ft=tex spell:
%
\href{mailto:johatfie@gmail.com}{johatfie@gmail.com}

        \\
        % linkedin.tex
% vim:set ft=tex spell:
%
\href{https://www.linkedin.com/in/jon-hatfield-abaab130/}{www.linkedin.com/in/jon-hatfield-abaab130/}
% \ \ \textbullet \ \ % github.tex
% vim:set ft=tex spell:
%
\href{https://github.com/johatfie}{github.com/johatfie}

        \\
    \else
        % address.tex
% vim:set ft=tex spell:
%
7679 Shasta Drive
\ \textbullet \ \ % city.tex
% vim:set ft=tex spell:
%
Indianapolis, IN
 % zip.tex
% vim:set ft=tex spell:
%
46217

        \\
        % phone.tex
% vim:set ft=tex spell:
%
(317) 513-0861
\ \textbullet \ \ % email.tex
% vim:set ft=tex spell:
%
\href{mailto:johatfie@gmail.com}{johatfie@gmail.com}
% \ \ \textbullet \ \
        \\
    \fi
    {\large Software Developer \ \ \textbullet \ \ \emph{"If I don't know it already, I'll learn it"}}
\end{center}

% headline.tex
% vim:set ft=tex spell:
%
\emph{Object-Oriented Analysis and Design \ | \ Scrum and Agile Methodologies \ | \ Design Patterns \ | \ Databases}

\\

\hrule
\vspace{-0.4em}
\subsection*{Profile}

% profile.tex
% vim:set ft=tex spell:
%
    \begin{itemize*}
        %\item Seeking a Java software developer position in the Indianapolis area
        \item Results-driven software developer with 6 years of experience developing software to meet ever-changing busi-
            ness requirements and finding innovative solutions to new and existing problems using Java, Ruby, and C++
        %\item Highly skilled at continuously improving code quality, improving test coverage, and refactoring code to improve maintainability
        %\item Known for finding automated solutions to existing manual processes saving much ongoing time and effort
        \item Adept at adopting new technologies and quickly becoming effective with them
        %\item Proficient with Java, SQL, Git, MongoDB, Perforce
    \end{itemize*}



\hrule
\vspace{-0.4em}
\subsection*{Experience}

\begin{itemize}
    \parskip=0.1em

    \begin{samepage}
    \item
        % stratice.tex
% vim:set ft=tex spell:
%
    \headerrow
        {\textbf{\href{https://www.straticehealthcare.com/}{Stratice Healthcare}}}
        {\textbf{Carmel, IN}}
    \\
    \headerrow
        {\emph{Web Developer, Contractor}}
        {\emph{2016 -- 2017}}
    \begin{itemize*}
        %\item Web Application Developer working in Java, Groovy, Grails, and AngularJS using Git to build two
            %web portals using a microservices architecture allowing doctors to electronically order durable medical equipment and
            %allowing suppliers to receive and fulfill those orders
        \item Web Application Developer providing enhancements, bug fixes, unit, and functional testing for two
            web portals in a microservices architecture using REST APIs allowing doctors to electronically prescribe durable medical equipment and
            allowing suppliers to receive and fulfill those prescriptions electronically
        \item Automated a lengthy series of manual UI tests for the supplier portal saving 3 man-days per two-week sprint
    \end{itemize*}

    \ifnum\pdfstrcmp{\jobname}{JonHatfieldPublicResume}=0
    \hspace{1.0em}
        {\textbf{Technologies:} Java, Groovy, G.r.a.i.l.s, REST, microservices, SQL, PostgreSQL, Hibernate, Scrum, IntelliJ, Git,
        A.n.g.u.l.a.r.J.S, Vim, Bash}
    \fi
    \ifnum\pdfstrcmp{\jobname}{JonHatfieldPublicResume_watermarked}=0
    \hspace{1.0em}
        {\textbf{Technologies:} Java, Groovy, G.r.a.i.l.s, REST, microservices, SQL, PostgreSQL, Hibernate, Scrum, IntelliJ, Git,
        A.n.g.u.l.a.r.J.S, Vim, Bash}
    \fi
    \ifnum\pdfstrcmp{\jobname}{JonHatfieldResume}=0
    \hspace{1.0em}
        {\textbf{Technologies:} Java, Groovy, Grails, REST API, microservices, SQL, PostgreSQL, Hibernate, Scrum, AngularJS, Vim, IntelliJ, Git, Bash}
    \fi
    %\hspace{1.0em}
        %{\textbf{Technologies:} Java, Groovy, Grails, REST, microservices, SQL, PostgreSQL, Hibernate, Scrum, AngularJS, Vim, IntelliJ, Git, Bash}

    \end{samepage}

    \begin{samepage}
    \item
        % salesforce.tex
% vim:set ft=tex spell:
%
    \headerrow
        {\textbf{\href{https://www.marketingcloud.com/}{Salesforce Marketing Cloud}}}
        {\textbf{Indianapolis, IN}}
    \\
    \headerrow
        {\emph{Software Engineer, Contractor}}
        {\emph{2016}}
    \begin{itemize*}
        \item Software Engineer on the Predictive Intelligence team adding and removing clients
            and making continuous improvements to 8 RESTful Microservices (REST API)
        \item Improved unit test code coverage by 20\%, and provided numerous updates to documentation
        \item Wrote over 70 AWK scripts for automatically correcting code style issues
    \end{itemize*}

    %\vspace{-1.2em}
    \ifnum\pdfstrcmp{\jobname}{JonHatfieldPublicResume}=0
        \hspace{1.0em}
        {\textbf{Technologies:} R.u.b.y, Bash, REST, microservices, Git, Docker, AWK, Vim, Scrum, R.u.b.y.M.i.n.e}
    \fi
    \ifnum\pdfstrcmp{\jobname}{JonHatfieldPublicResume_watermarked}=0
        \hspace{1.0em}
        {\textbf{Technologies:} R.u.b.y, Bash, REST, microservices, Git, Docker, AWK, Vim, Scrum, R.u.b.y.M.i.n.e}
    \fi
    \ifnum\pdfstrcmp{\jobname}{JonHatfieldResume}=0
        \hspace{1.0em}
        {\textbf{Technologies:} Ruby, Bash, REST API, microservices, Git, Docker, AWK, Vim, Scrum, RubyMine}
    \fi
    %\hspace{1.0em}
    %{\textbf{Technologies:} Ruby, Bash, REST, microservices, Git, Docker, AWK, Vim, Scrum, RubyMine}

    \end{samepage}

    \begin{samepage}
    \item
        % interactive.tex
% vim:set ft=tex spell:
%
    \headerrow
        {\textbf{\href{https://www.genesys.com/inin}{Interactive Intelligence}}}
        {\textbf{Indianapolis, IN}}
    \\
    \headerrow
        {\emph{Software Engineer}}
        {\emph{4 months}}
    \begin{itemize*}
        %\item On the Build team developing in Ruby, Java, and C++ to maintain the development environment for developers,
            %to add new capabilities, and to automate manual processes
         \item Designed and built a multi-threaded Java server-client application for extracting strings from
                source code files and submitting them to Acrolinx application for style, spelling, grammar, and terminology
                usage checks
    \end{itemize*}

    %\vspace{-1.2em}
    \hspace{1.0em}
    %\textbf{Technologies:} Ruby, Java, C++, Vim, MongoDB, RubyMine, Perforce, Jenkins
    \textbf{Technologies:} Java, Vim, MongoDB, Perforce, Ruby
    %\vspace{1.0em}


    \end{samepage}

    \begin{samepage}
    \item
        % searchsoft.tex
% vim:set ft=tex spell:
%
    \headerrow
        {\textbf{\href{http://www.searchsoft.net/}{SearchSoft Solutions}}}
        {\textbf{Indianapolis, IN}}
    \\
    \headerrow
        {\emph{Java Developer}}
        {\emph{2011 -- 2012}}
    \begin{itemize*}
        \item Java developer delivering new features, modules, and ongoing maintenance of Job Applicant web portal
            helping K-12 schools find qualified teachers and staff
        \item Redesigned and rebuilt custom backup solution in Ruby for backing up servers, PC’s, and databases
            across the LAN
        %\\
        %\textbf{Technologies:} Java, Eclipse, SQL Server, Ruby, Subversion
    \end{itemize*}

    %\vspace{-1.2em}
    \hspace{1.0em}
    \textbf{Technologies:} Java, Eclipse, SQL, Ruby, Subversion, SQL Server, Scrum

    \end{samepage}
\end{itemize}


%\begin{samepage}
%\hrule
%\vspace{-0.4em}
%\subsection*{Accomplishments}

%% accomplishments.tex
% vim:set ft=tex spell:
%
    \begin{itemize*}
        %\item Designed and built a multi-threaded Java server application with Vim for extracting strings from source code
                %files and submitting them to Acrolinx for style, spelling, grammar, and terminology usage checks
        \item Designed and built a matrix template class in C++ capable of matrix arithmetic and linear algebra using the
                valarray class from the C++ Standard Template Library
        \item Implemented Myers-Briggs personality test web app based on PHP, Apache, and MySQL using Dreamweaver~8
        \item Designed and built GUI-based, server-client model Forest Experiment database using SQL in SQL Server 2008
                using ERwin Data Modeler and Microsoft Visual Studio
        \item Designed and built simulated cell phone contacts ``fast find'' feature in Java using Java Swing Components
                with Net Beans IDE
        \item Designed and built Blackjack game with a 5 deck shoe in C++ using Microsoft Visual Studio
    \end{itemize*}

%\end{samepage}


\begin{samepage}
\hrule
\vspace{-0.4em}
\subsection*{Education}

% education.tex
% vim:set ft=tex spell:
%
\begin{itemize}
    \parskip=0.1em

    \item
    \headerrow
        {\textbf{\href{https://www.iupui.edu/}{Indiana University-Purdue University Indianapolis}}}
        {\textbf{Indianapolis, IN}}
    \\
    \headerrow
        {\emph{Purdue School of Science, B.S. Computer Science}}
        {\emph{2008 -- 2011}}
    \headerrow
        {\emph{Purdue School of Science, B.S. Applied Mathematics}}
        {\emph{2008 -- 2011}}
\end{itemize}

\end{samepage}


%\begin{samepage}
%\hrule
%\vspace{-0.4em}
%\subsection*{Core Technical Skills}

%\begin{indentsection}{\parindent}
%\hyphenpenalty=1000
%\begin{description*}
    %\item[Languages:]
    %C, \CPP, Java, JavaScript, \LaTeX, Perl, Python, shell script, SQL
    %\item[Open Source Contributions:]
    %Project A, Project B, Project C
%\end{description*}
%\end{indentsection}
%\end{samepage}

\end{document}
